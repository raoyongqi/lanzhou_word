% !TEX TS-program = xelatex
% !TEX encoding = UTF-8 Unicode

% \documentclass[AutoFakeBold]{LZUThesis}
\documentclass[AutoFakeBold]{LZUThesis-PgD&PhD}


\begin{document}
%=====%
%
%封皮页填写内容
%
%=====%



\schoolcode{10730}
\secret{公开}
\cid{025200}
% \yjsType{博士}
\yjsType{硕士}

% \yjsZsZy{\quad 学\quad 术\quad 学\quad 位\quad}
\yjsZsZy{\quad 专\quad 业\quad 学\quad 位\quad}


% 标题样式 使用 \title{{}}; 使用时必须保证至少两个外侧括号
%  如: 短标题 \title{{第一行}},
% 	      长标题 \title{{第一行}{第二行}}
%             超长标题\tiitle{{第一行}{...}{第N行}}
\title{{基于机器学习的植物病害分析与预测模型构建}}

% 标题样式 使用 \entitle{{}}; 使用时必须保证至少两个外侧括号
%  如: 短标题 \entitle{{First row}},
% 	      长标题 \entitle{{First row}{ Second row}}
%             超长标题\entitle{{First row}{...}{ Next N row}}
% 注意:  英文标题多行时 需要在开头加个空格 防止摘要标题处英语单词粘连。
\entitle{
         { Construction of Machine Learning-Based Models}
         { for Plant Disease Analysis and Prediction}}

\author{饶永祺}

% \major{一级学科·专业}
\major{应用统计}

\research{生态统计学}

% \education{学历教育/同等学力人员申请博士学位}
\education{学历教育}
% \education{学历教育/同等学力人员申请硕士学位/在职攻读硕士专业学位(非学历)}

\advisor{刘向 研究员}
\codvisor{} %合作导师,可为空,但不可没有这一栏
\elapse{2022 年 9月\quad 至 \quad 2025 年 6 月}
\defense{2025 年 6 月}

\maketitle

%======%
%诚信说明页
%授权说明书
%======%
% 如果超出边界,可以调整签字的宽度,现在是50,如果你不用,把下面的注释就好

% 你的签名
\mysignature{
    % \raisebox{-5pt}{
    % \includegraphics[width=40pt]{signature.pdf}
    % }
}
% 你手写的日期
\mytime{
    % \raisebox{-5pt}{
    % \includegraphics[width=40pt]{signature.pdf}
    % }
}
% 老师的手写签名
\supervisorsignature{
    % \raisebox{-5pt}{
    % \includegraphics[width=40pt]{signature.pdf}
    % }
}
% 老师手写的时间
\teachertime{
    % \raisebox{-5pSt}{
    % \includegraphics[width=40pt]{signature.pdf}
    % }
}
% 老师手写的成绩
\recommendedgrade{
    % \raisebox{-5pt}{
    % \includegraphics[width=40pt]{signature.pdf}
    % }
}

\makestatement


\frontmatter



%中文摘要
\ZhAbstract{植物病害是影响农业生产的主要因素之一,它们不仅造成作物产量的减少,还可能引发食品安全问题。随着全球气候变化的加剧和农田管理的挑战,传统的植物病害防治方法面临许多困难。因此,开发高效、精确的植物病害分析与预测方法显得尤为重要。近年来,集成学习(Ensemble Learning)技术凭借其强大的预测能力,成为植物病害分析与预测领域的重要工具,尤其是在大尺度区域的应用中。集成学习通过将多个基础学习器的预测结果结合起来,从而提高模型的稳定性和准确性,特别适用于处理大范围、复杂的农业环境数据。

在中国,农业生产覆盖广阔的地理区域,植物病害的发生具有显著的空间和时间变化特征。传统的植物病害预测方法依赖于人工观察和诊断,这种方法既费时又容易受到人为因素的影响。此外,植物病害的症状通常具有相似性,且在不同气候和环境条件下表现不一,传统方法的准确性和适应性有限。尽管遥感技术和传感器技术的进步提高了病害检测的精度,但这些技术通常依赖大量人工干预,难以实现自动化和实时监控,特别是在广泛区域的农业生产中。

集成学习方法能够有效应对这一挑战。常见的集成学习算法如随机森林(Random Forest, RF)、提升树(Boosting)方法如XGBoost和LightGBM,以及袋装(Bagging)方法如Adaboost,在植物病害预测中均取得了显著成果。这些算法通过结合多个弱学习器的预测结果,提高了模型在复杂农业数据环境中的稳定性和准确性。通过融合来自不同来源的大尺度数据,如气象数据、土壤数据、农田管理信息等,集成学习能够识别并提取出与病害发生相关的关键特征,从而进行精准的预测。

在中国,气候变化、季节性降水、土壤条件等因素的多样性使得植物病害的发生呈现区域性和时间性差异。为了提高病害预测的准确性,集成学习方法特别适合处理这些大尺度和多维度的数据。气象数据(如温度、湿度、降水量)和土壤数据(如土壤湿度、pH值、养分含量)为集成学习模型提供了重要的输入变量。此外,农作物生长季节的变化和不同地理区域的病害历史数据也能够作为训练集的重要组成部分,通过数据融合,集成学习能够准确捕捉到病害的潜在规律。

集成学习模型在训练过程中,能够自动学习和优化来自不同数据源的信息,并对不同特征赋予适当的权重。通过这种方式,集成学习模型能够克服单一模型的局限性,提高对复杂数据模式的识别能力。在植物病害预测中,集成学习方法不仅能够处理空间数据,还能够结合时间序列数据,如历史病害数据和气象预测数据,进行时空分析,进一步提高预测精度。

然而,尽管集成学习在植物病害预测中取得了良好效果,仍然面临一些挑战。首先,数据的质量和数量是集成学习成功应用的关键。由于植物病害发生的复杂性,现有的数据往往存在噪声,并且缺乏全面性和代表性,这可能影响模型的训练效果。其次,集成学习模型的训练过程通常需要较长时间和较高的计算资源,尤其是在大尺度区域的应用中,需要处理大量的地理和气象数据。最后,集成学习模型的可解释性问题仍未得到完全解决,农业领域的从业人员需要能够理解模型的预测结果,以便做出合理的决策。

随着计算能力和数据采集技术的进步,未来集成学习在植物病害预测中的应用将越来越广泛。未来的研究应进一步注重多模态数据的融合,通过结合气象数据、土壤数据、农作物信息以及历史病害数据,提升模型的准确性和鲁棒性。此外,考虑到不同地区病害的差异性,集成学习模型应具备自适应学习能力,能够根据不同区域的环境和气候条件进行调整,以实现跨地区、跨作物的预测能力。

总之,集成学习在植物病害分析与预测中展现了巨大的应用潜力。特别是在中国这样的大规模农业生产背景下,集成学习能够利用丰富的气象、土壤和农作物生长数据,为农民提供及时、准确的病害预警和防治方案。随着技术的发展和数据的进一步积累,集成学习将在促进精准农业和提高农作物产量方面发挥更加重要的作用。
}{集成学习,大尺度研究}


%英文摘要
\EnAbstract{Plant diseases are one of the major factors affecting agricultural production. They not only lead to a reduction in crop yield but can also cause food safety issues. With the intensification of global climate change and the challenges of farmland management, traditional plant disease control methods face numerous difficulties. Therefore, it is crucial to develop efficient and accurate methods for plant disease analysis and prediction. In recent years, ensemble learning techniques, with their powerful predictive capabilities, have become important tools in the field of plant disease analysis and prediction, especially in large-scale applications. Ensemble learning combines the prediction results of multiple base learners to improve the model's stability and accuracy, making it particularly suitable for handling large-scale, complex agricultural environmental data.

In China, agricultural production covers vast geographic areas, and the occurrence of plant diseases exhibits significant spatial and temporal variability. Traditional plant disease prediction methods rely on manual observation and diagnosis, which are time-consuming and prone to human error. In addition, plant disease symptoms often have similarities and vary under different climatic and environmental conditions, limiting the accuracy and adaptability of traditional methods. Although the advancements in remote sensing and sensor technologies have improved disease detection accuracy, these technologies usually rely on significant human intervention, making it difficult to achieve automation and real-time monitoring, especially in large-scale agricultural production areas.

Ensemble learning methods can effectively address this challenge. Common ensemble learning algorithms such as Random Forest (RF), boosting methods like XGBoost and LightGBM, and bagging methods like AdaBoost, have all achieved significant results in plant disease prediction. These algorithms improve model stability and accuracy in complex agricultural data environments by combining the prediction results of multiple weak learners. By integrating large-scale data from various sources, such as meteorological data, soil data, and farm management information, ensemble learning can identify and extract key features related to disease occurrence, enabling precise predictions.

In China, the diversity of factors such as climate change, seasonal rainfall, and soil conditions leads to regional and temporal variations in plant disease occurrence. To improve disease prediction accuracy, ensemble learning methods are particularly suited for handling such large-scale and multidimensional data. Meteorological data (e.g., temperature, humidity, precipitation) and soil data (e.g., soil moisture, pH, nutrient content) provide important input variables for ensemble learning models. In addition, changes in crop growth seasons and historical disease data from different geographic regions can also serve as important components of the training set. Through data fusion, ensemble learning can accurately capture the underlying patterns of disease occurrence.

During the training process, ensemble learning models can automatically learn and optimize information from different data sources, assigning appropriate weights to different features. In this way, ensemble learning models can overcome the limitations of single models and improve their ability to recognize complex data patterns. In plant disease prediction, ensemble learning methods can process spatial data and also integrate time-series data, such as historical disease data and meteorological forecast data, to perform spatiotemporal analysis, further improving prediction accuracy.

However, despite the positive results of ensemble learning in plant disease prediction, there are still some challenges. First, the quality and quantity of data are critical to the successful application of ensemble learning. Due to the complexity of plant disease occurrence, the existing data often contains noise and lacks comprehensiveness and representativeness, which may affect model training effectiveness. Second, the training process of ensemble learning models typically requires a long time and considerable computational resources, especially when applied on a large scale, as it involves processing massive geographical and meteorological data. Finally, the interpretability of ensemble learning models has not been fully resolved, and agricultural practitioners need to understand the model's predictions to make informed decisions.

With advancements in computing power and data acquisition technology, the application of ensemble learning in plant disease prediction will become more widespread in the future. Future research should focus more on multimodal data fusion, by combining meteorological data, soil data, crop information, and historical disease data, to enhance the accuracy and robustness of the models. Additionally, given the variability of diseases in different regions, ensemble learning models should have adaptive learning capabilities, allowing them to adjust according to the environmental and climatic conditions of different regions, thereby achieving cross-regional and cross-crop prediction capabilities.

In conclusion, ensemble learning demonstrates enormous potential in plant disease analysis and prediction. Particularly in large-scale agricultural production contexts like China, ensemble learning can leverage rich meteorological, soil, and crop growth data to provide timely and accurate disease warnings and control strategies for farmers. With the development of technology and the further accumulation of data, ensemble learning will play an increasingly important role in promoting precision agriculture and improving crop yields.
    % \fontspec{Times New Roman} {Times New Roman}
}{Ensemble Learning, Large-Scale Research}

%生成目录
% \tableofcontents
% 下面这个包含图表目录
\customcontent

% % 部分同学需要专业术语注释表,* 表示不加入目录
% \chapter*{专业术语注释表}
% \begin{longtable}{lll}
%   \caption*{缩略词说明}\\
%   SS & Spread Spectrum & 扩展频谱 \\
%   PAPR & Peak to Average Power Ratio & 峰均比\\
%   DCSK & Differential Chaos Shift Keying &差分混移位键控\\
%   dasd & fdhfudw eqwrqw fasfasfs fewev wqfwefew &\tabincell{l}{太长了\\换行一下}\\
% \end{longtable}

%文章主体
\mainmatter

\chapter{注意事项}

\section{编译方式}

 {\bfseries 编译方式:} XeLaTeX -->BibTeX --> XeLaTeX-->XeLaTeX

如果你和我一样使用Atom编辑器,在配置好latex环境后,选择 文件--> 设置 --> 拓展 --> latex --> 设置 --> Engine,修改为XeLatex即可。

\section{插入图片}

\begin{figure}[hbt!]
    \includegraphics[width=0.95\textwidth]{example-image-a}
    \centering
    \caption{示例图片A:多点字多点字多点字多点字多点字多点字多点字多点字多点字多点字多点字多点字多点字多点字多点字多点字多点字多点字多点字多点字多点字多点字多点字多点字多点字多点字多点字多点字多点字多点字多点字多点字多点字多点字多点字多点字多点字多点字多点字多点字多点字多点字多点字多点字多点字}
    \label{fig:obj_dect}
\end{figure}


\begin{figure}[H]
    \centering
    \subfloat[111]{
        \includegraphics[width=0.3\textwidth]{figures/lzu2007.png}
    }
    \subfloat[2222]{
        \includegraphics[width=0.3\textwidth]{figures/lzu2020.png}
    }
    \caption{两个图:多点字多点字多点字多点字多点字多点字多点字多点字多点字多点字多点字多点字多点字多点字多点字多点字多点字多点字多点字多点字多点字多点字多点字多点字多点字多点字多点字多点字多点字多点字多点字多点字多点字多点字多点字多点字多点字多点字多点字多点字多点字多点字多点字多点字多点字}
    \label{fig_ldr}
\end{figure}

尽管对论文图片的大小没有具体的规定,但还是建议插入可以横向占满可写宽度的图片比较好看,一个示例如上图:

% \begin{verbatim}
% \begin{figure}[hbt!]
%   \includegraphics[width=0.95\textwidth]{example-image-a}
%   \centering
%   \caption{示例图片A}
%   \label{fig:example-a}
% \end{figure}
% \end{verbatim}



\section{公式}

\subsection{一般公式}
\subsubsection{一般公式22}

一般公式手敲即可。

% \begin{verbatim}
% \begin{equation}\label{eq:sip}
%   a_1 + b_2 = 3
% \end{equation}
% \end{verbatim}


\begin{equation}\label{eq:sip}
    a_1 + b_2 = 3
\end{equation}


\subsection{多行公式}

多行公式,对于不想要标号的部分,可以使用nonumber进行标注:

% \begin{verbatim}
% \begin{gather}\label{eq:add}
% 1+1=2 \\
% 2+2=4 \\
% 3+3=6 \nonumber
% \end{gather}
% \end{verbatim}

\begin{gather}\label{eq:add}
    1+1=2 \\
    2+2=4 \\
    3+3=6 \nonumber
\end{gather}

\subsection{多情况公式}
带括号的多种情况的公式,其中X为数学粗体。

% \begin{verbatim}
% \begin{equation}\label{eq:multi}
%   \mathbf{X}=
%     \begin{cases}
%       k_n \quad & n \ = \ 1 \\
%       \mathbf{X}_n \ = \mathbf{X}_{n-1}\ +\ (k_n-1)\times S_{n-1};
%       \quad & n \geq \ 1\\
%     \end{cases}
% \end{equation}
% \end{verbatim}

\begin{equation}\label{eq:multi}
    \mathbf{X}=
    \begin{cases}
        k_n \quad & n \ = \ 1  \\
        \mathbf{X}_n \ = \mathbf{X}_{n-1}\ +\ (k_n-1)\times S_{n-1};
        \quad     & n \geq \ 1 \\
    \end{cases}
\end{equation}

\subsection{公式加粗、斜体、字体}

公式、字母加粗、字体问题

\begin{itemize}
    \item[1.] 正文 \qquad \quad AHEMoS$\alpha \beta$
    \item[2.] 公式 \qquad \quad $AHEMoS \alpha \beta$
    \item[3.] mathbf \qquad $\mathbf{AHEMoS\alpha \beta}$
    \item[4.] boldsymbol $\boldsymbol{AHEMoS\alpha \beta}$
    \item[5.] mathbb \qquad $\mathbb{AHEMoS\alpha \beta}$
          % \item [5. bm] $\bm{AHEMoS\alpha \beta}$
\end{itemize}

这个加粗、斜体、英文字体(含正文和公式内字体),有不同的处理方式,在 .cls 模板文件文件搜索 bm 查看详细说明

\subsection{一些特殊符号}

\begin{itemize}
    \item 普朗克常量 hslash :$\hslash$
    \item 普朗克常量 hbar :$\hbar$
    \item 花体 mathscr :$\mathscr{ABCFR}$
    \item 花体 mathcal :$\mathcal{ABCFR}$
    \item Fraktur字母 :$\mathfrak{ABCFR}$
\end{itemize}


\section{表格}

三线表格式:

测试测试测试测测试测试测试测试测试测试测试测试测测试测试测试测试测试测试测试测试测测试测试测试测试测试测试测试测试测测试测试测试测试测试测试测试测试测测试测试测试测试测试

% \begin{verbatim}
% \begin{table}[hbt!]\label{tbl:mole}
%   \centering
%   \begin{tabular*}{0.9\textwidth}{@{\extracolsep{\fill}}cccccc}
%     \toprule
%         参数& m & n & \tabincell{c}{太长了\\换行一下\\原子数}  & 内径 & 长度\\
%     \midrule
%     数值 & 15 & 15  & 2880 & 2.3014nm & 9.95nm \\
%     \bottomrule
%   \end{tabular*}
%   \caption{table example}
% \end{table}
% \end{verbatim}

\begin{table}[hbt!]\label{tbl:mole}
    \centering
    \begin{tabular*}{0.9\textwidth}{@{\extracolsep{\fill}}cccccc}
        \toprule
        参数& m & n & \tabincell{c}{太长了\\换行一下\\原子数}  & 内径 & 长度\\
        \midrule
        数值 & 15 & 15  & 2880 & 2.3014nm & 9.95nm \\
        \bottomrule
    \end{tabular*}
    \caption{table example}
\end{table}


\section{引用}

\subsection{论文引用}

建议以 web of science 或者文献官网导出的bib为准,请不要使用百度学术、谷歌学术的bib,错误很多,\cite{partl2016, tenne1992polyhedral, tussyadiah2015hotels}。
测试不同情况:

测试不同情况:$\mathcal{P}\mathcal{T}$ 就是宋敏\songti{玥}这个玥编译不出来

\begin{itemize}
    \item 原本科模板\cite{partl2016}
    \item 中文“等”测试\cite{partl2021}
    \item 大写字母测试\cite{partl2022-2}
    \item 连接符号测试\cite{partl2022-3}
    \item 中文空格测试\cite{partl2022}
    \item 连续显示\cite{partl2021,partl2022-2,partl2022-3}
    \item 右上角\cite{partl2016,partl2021,partl2022-2}
    \item 中文参考文献 \cite{李刚2006基于动态光谱的脉搏血氧测量精度分析}
    \item 标题中特殊符号,bib中双层大括号即可 \cite{PhysRevLett.108.024101}
\end{itemize}

\section{图表引用}

% \begin{verbatim}
% 如\hyperref[tbl:mole]{表 1-1}所示。
% \end{verbatim}

使用label和hyperref进行引用,引用时用图标的意义命名,尽量少用类似tbl:3-1这样,而是eq:sum-up这样有意义的,如\hyperref[tbl:mole]{表 1-1}所示。

\section{其他}


伪代码

\begin{algorithm}[H]
    \caption{PMHSS 算法\label{Alg:PMHSS}}
    \begin{algorithmic}[1]
        \State 给定一个初值 $ x^{(0)} \in C^{n} $  和常数 $\alpha>0$
        \For{$k = 1,2, \ldots $ 直到序列 $\{x^{(k)}\}_{k=0}^{\infty}$ 收敛}
        \State 解方程: $(\alpha V+W)x^{(k+\frac{1}{2})}=(\alpha V-i T)x^{(k)}+b $
        \State 解方程: $(\alpha V+T)x^{(k+1)}=(\alpha V+i W)x^{(k+\frac{1}{2})}-i b$
        \EndFor
    \end{algorithmic}
\end{algorithm}

使用\textbackslash blank 来空行,\textbackslash blackpage 来空白页。
空行用在在校成果罗列,空白页用来补充双页打印留白。

无blank

\blank

间隔blank
\blankpage




%论文后部
\backmatter


%=======%
%引入参考文献文件
%=======%
\bibdatabase{bib/database}%bib文件名称 仅修改bib/ 后部分
\printbib
% \nocite{*} %显示数据库中有的,但是正文没有引用的文献



\Achievements


\blank




\Thanks

感谢刘向老师三年以来的指导和培养,感谢父母在学费,生活费给与的支持,感谢同班同学,师兄师姐的关心与帮助。

\end{document}
